\documentclass{article}
\usepackage[colorlinks]{hyperref}

\usepackage[a4paper, landscape, margin=0.5cm]{geometry}
\usepackage{multicol}

%--------------------------------------
\usepackage[T1]{fontenc}
\usepackage[utf8]{inputenc}
%--------------------------------------

%Portuguese-specific commands
%--------------------------------------
\usepackage[portuguese]{babel}
%--------------------------------------

% ver nomes das cores em:
% https://www.overleaf.com/learn/latex/Using_colours_in_LaTeX#Further_reading
% svgnames
\usepackage[svgnames]{xcolor}
\usepackage{framed,graphicx}
\usepackage{amsthm}
% macro de destaque - pegar do template latex
\usepackage{fancyvrb}
\newcommand{\VerbBar}{|}
\newcommand{\VERB}{\Verb[commandchars=\\\{\}]}
\DefineVerbatimEnvironment{Highlighting}{Verbatim}{commandchars=\\\{\}}
% Add ',fontsize=\small' for more characters per line
\newenvironment{Shaded}{}{}
\newcommand{\AlertTok}[1]{\textcolor[rgb]{1.00,0.00,0.00}{\textbf{#1}}}
\newcommand{\AnnotationTok}[1]{\textcolor[rgb]{0.38,0.63,0.69}{\textbf{\textit{#1}}}}
\newcommand{\AttributeTok}[1]{\textcolor[rgb]{0.49,0.56,0.16}{#1}}
\newcommand{\BaseNTok}[1]{\textcolor[rgb]{0.25,0.63,0.44}{#1}}
\newcommand{\BuiltInTok}[1]{#1}
\newcommand{\CharTok}[1]{\textcolor[rgb]{0.25,0.44,0.63}{#1}}
\newcommand{\CommentTok}[1]{\textcolor[rgb]{0.38,0.63,0.69}{\textit{#1}}}
\newcommand{\CommentVarTok}[1]{\textcolor[rgb]{0.38,0.63,0.69}{\textbf{\textit{#1}}}}
\newcommand{\ConstantTok}[1]{\textcolor[rgb]{0.53,0.00,0.00}{#1}}
\newcommand{\ControlFlowTok}[1]{\textcolor[rgb]{0.00,0.44,0.13}{\textbf{#1}}}
\newcommand{\DataTypeTok}[1]{\textcolor[rgb]{0.56,0.13,0.00}{#1}}
\newcommand{\DecValTok}[1]{\textcolor[rgb]{0.25,0.63,0.44}{#1}}
\newcommand{\DocumentationTok}[1]{\textcolor[rgb]{0.73,0.13,0.13}{\textit{#1}}}
\newcommand{\ErrorTok}[1]{\textcolor[rgb]{1.00,0.00,0.00}{\textbf{#1}}}
\newcommand{\ExtensionTok}[1]{#1}
\newcommand{\FloatTok}[1]{\textcolor[rgb]{0.25,0.63,0.44}{#1}}
\newcommand{\FunctionTok}[1]{\textcolor[rgb]{0.02,0.16,0.49}{#1}}
\newcommand{\ImportTok}[1]{#1}
\newcommand{\InformationTok}[1]{\textcolor[rgb]{0.38,0.63,0.69}{\textbf{\textit{#1}}}}
\newcommand{\KeywordTok}[1]{\textcolor[rgb]{0.00,0.44,0.13}{\textbf{#1}}}
\newcommand{\NormalTok}[1]{#1}
\newcommand{\OperatorTok}[1]{\textcolor[rgb]{0.40,0.40,0.40}{#1}}
\newcommand{\OtherTok}[1]{\textcolor[rgb]{0.00,0.44,0.13}{#1}}
\newcommand{\PreprocessorTok}[1]{\textcolor[rgb]{0.74,0.48,0.00}{#1}}
\newcommand{\RegionMarkerTok}[1]{#1}
\newcommand{\SpecialCharTok}[1]{\textcolor[rgb]{0.25,0.44,0.63}{#1}}
\newcommand{\SpecialStringTok}[1]{\textcolor[rgb]{0.73,0.40,0.53}{#1}}
\newcommand{\StringTok}[1]{\textcolor[rgb]{0.25,0.44,0.63}{#1}}
\newcommand{\VariableTok}[1]{\textcolor[rgb]{0.10,0.09,0.49}{#1}}
\newcommand{\VerbatimStringTok}[1]{\textcolor[rgb]{0.25,0.44,0.63}{#1}}
\newcommand{\WarningTok}[1]{\textcolor[rgb]{0.38,0.63,0.69}{\textbf{\textit{#1}}}}
% https://tex.stackexchange.com/questions/64298/error-with-xcolor-package


\setcounter{secnumdepth}{0} % removendo numeros
\usepackage{titlesec}
\usepackage{sectsty}
\setlength{\parindent}{0em} % corrigindo identação


% Melhorando as descrições - http://eikimartinson.com/archives/90-LaTeX-Description-Lists-with-Dot-Leaders.html
\usepackage{enumitem}
%\setlist[description,hlevelsi]{hformati}
\setlist[description]{style=sameline}
\setlist{noitemsep}


\usepackage{mdframed}[framemethod=tikz]
\mdfdefinestyle{MeuEstilo}{linecolor=blue,frametitle=Nice,rightline=false,leftline=false}

\begin{document}




\section{Qualquer coisa}

\begin{multicols*}{4}


%\subsectionfont{\color{green}}

\colorlet{shadecolor}{green!10}

\begin{leftbar}
\hypertarget{tipos-de-dados}{%
\subsection*{Tipos de dados}\label{tipos-de-dados}}


%\color{DarkGreen}


\texttt{Integers} \ldots{} 10

\texttt{Float} \ldots{} 3.14

\texttt{String} \ldots{} ``abracadabra'', `magia'

\texttt{Booleans} \ldots{} \texttt{True}, \texttt{False}

\end{leftbar}

\begin{mdframed}[backgroundcolor=gray!20]
\texttt{Integers} \hfill 10

\texttt{Float} \dotfill 3.14

\texttt{String} \hrulefill ``abracadabra'', `magia'

\texttt{Booleans} \hfill \texttt{True}, \texttt{False}

\end{mdframed}

\subsectionfont{\color{Grey}}

\subsection*{Outra seção}

\color{DarkGrey}

Conteúdo aqui.


\texttt{Integers} \ldots{} 10

\texttt{Float} \ldots{} 3.14

\texttt{String} \ldots{} ``abracadabra'', `magia'

\texttt{Booleans} \ldots{} \texttt{True}, \texttt{False}

Abrindo arquivos:

\begin{Shaded}
\begin{Highlighting}[]
\ControlFlowTok{if}\NormalTok{ x}\OperatorTok{>}\DecValTok{3}\NormalTok{::}
   \ControlFlowTok{return}\NormalTok{ false}
\end{Highlighting}
\end{Shaded}

\texttt{Integers} \hfill 10

\texttt{Float} \dotfill 3.14

\texttt{String} \hrulefill ``abracadabra'', `mágica'


\section{Mais coisas}

The limerick packs laughs anatomical\\
In space that is quite economical.\\
\hspace*{0.333em}\hspace*{0.333em}\hspace*{0.333em}But the good ones
I've seen\\
\hspace*{0.333em}\hspace*{0.333em}\hspace*{0.333em}So seldom are clean\\
And the clean ones so seldom are comical

200 Main St.\\
Berkeley, CA 94718


\begin{description}
\item[Term 1]
Definition 1
\item[Term 2 with \emph{inline markup}]
Definition 2

\begin{verbatim}
{ some code, part of Definition 2 }
\end{verbatim}

Third paragraph of definition 2.
\end{description}

\begin{center}\rule{0.5\linewidth}{\linethickness}\end{center}



\subsectionfont{\color{red}}

\subsection{Texto qualquer}

\color{DarkRed}

\begin{description}
\item[Term 1]
Definition 1
\item[Term 2 with \emph{inline markup}]
Definition 2

\begin{verbatim}
{ some code, part of Definition 2 }
\end{verbatim}

Third paragraph of definition 2.
\end{description}



\begin{mdframed}[style=MeuEstilo]

O que mais?

\begin{description}
\item[Term 1]
Definition 1
\item[Term 2 with \emph{inline markup}]
Definition 2

\begin{verbatim}
{ some code, part of Definition 2 }
\end{verbatim}

Third paragraph of definition 2.
\end{description}


In any right triangle, the area of the square whose side is the hypotenuse
is equal to the sum of the areas of the squares whose sides are the two
legs.
\end{mdframed}

\end{multicols*}




\end{document}
